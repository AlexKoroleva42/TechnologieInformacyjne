\documentclass[a4paper,12pt]{article}
\usepackage[utf8]{inputenc}

\usepackage[12pt]{extsizes}

\usepackage[left=2cm,top=1cm,right=2cm,bindinoffset=0cm]{geometry}

\usepackage[T2A]{fontenc}
\usepackage[utf8]{inputenc}
\usepackage[english,polish]{babel}

\usepackage{amsmath, amsfonts, amssymb, amsthm, mathtools, dsfont}

\title{LaTeX Lab4}
\author{Timofey Taushanau}
\date{Grudzień 2022}\documentclass[a4paper,12pt]{article}
\usepackage[utf8]{inputenc}

\usepackage[12pt]{extsizes}

\usepackage[left=2cm,top=1cm,right=2cm,bindinoffset=0cm]{geometry}

\usepackage[T2A]{fontenc}
\usepackage[utf8]{inputenc}
\usepackage[english,polish]{babel}

\usepackage{amsmath, amsfonts, amssymb, amsthm, mathtools, dsfont}
\usepackage[T1]{fontenc}
\usepackage{pgfplots}
\pgfplotsset{width=15cm,compat=1.9}

% We will externalize the figures
\usepgfplotslibrary{external}
\tikzexternalize

\pgfplotsset{
  /pgfplots/error bar legend/.style={
    legend image code/.code={
        \draw[sharp plot,mark=-,mark repeat=2,mark phase=1,color=gray,##1]
        plot coordinates { (0.3cm, -0.15cm) (0.3cm,0cm) (0.3cm, 0.15cm) };%
        \draw[mark repeat=2,mark phase=2,##1]
        plot coordinates {(0cm,0cm) (0.3cm,0cm) (0.6cm,0cm)};%
        }}}

\title{LaTeX Lab4}
\author{Timofey Taushanau}
\date{Grudzień 2022}

\usepackage{graphicx}
\graphicspath{ {./images/} }
\usepackage[T1]{fontenc}


\begin{document}

\maketitle

\section{Wprowadzenie teoretyczne}

Dla ciała, spadającego swobodnie, kinematyczne równanie będzie przedstawione w następnej postaci:

\begin{equation}\label{Kinematyka}
    h(t) = h_0 + \frac{gt^2}{2}
\end{equation}

gdzie $h_0$ – początkowa wysokość, $g$ - przyspieszenie ziemskie.

Z równania \eqref{Kinematyka} wynika, że:
\begin{equation}\label{Przyspieszenie}
    g = \frac{2h_0}{t^2}
\end{equation}
Bo po spadnięciu ciała, wysokość w ostatnim momencie będzie równać się 0.

\section{Opis eksperymentu}

\begin{figure}[h]
    \centering
    \includegraphics{spadek}
    \caption{Przykład spadka swobodnego}
    \label{fig:rys1}
\end{figure}

Jak widać na rysunku \ref{fig:rys1}, będziemy wyrzucać ciało ze stałej wysokości kilku razy i zmierzać czas spadania. Znając początkową wysokość i czas spadania, możemy łatwo obliczyć przyspieszenie ziemskie.

\section{Wyniki pomiarów}

\begin{table}[b]
\centering
\begin{tabular}{|c|c|c|}
        \hline
        $t[s]$ & $s[m]$ & $s+\Delta s[m]$ \\
        \hline
             0&     0&  0.016\\
        \hline
        0.100&	0.049&	0.052\\
        \hline
        0.200&	0.196&	0.196\\
        \hline
        0.300&	0.441&	0.440\\
        \hline
        0.400&	0.784&	0.796\\
        \hline
        0.500&	1.225&	1.235\\
        \hline
        0.600&	1.764&	1.753\\
        \hline
        0.700&	2.401&	2.389\\
        \hline
        0.800&	3.136&	3.126\\
        \hline
        0.900&	3.969&	3.982\\
        \hline
        1.000&	4.900&	4.891\\
        \hline
        1.100&	5.929&	5.911\\
        \hline
        1.200&	7.056&	7.055\\
        \hline
        1.300&	8.281&	8.291\\
        \hline
        1.400&	9.604&	9.610\\
        \hline
        1.500&	11.025&	11.022\\
        \hline
        1.600&	12.544&	12.551\\
        \hline
        1.700&	14.161&	14.156\\
        \hline
        1.800&	15.876&	15.867\\
        \hline
        1.900&	17.689&	17.680\\
        \hline
        2.000&	19.600&	19.602\\
        \hline
        2.100&	21.609&	21.605\\
        \hline
        2.200&	23.716&	23.705\\
        \hline
        2.300&	25.921&	25.930\\
        \hline
        2.400&	28.224&	28.233\\
        \hline
        2.500&	30.625&	30.630\\
        \hline
        2.600&	33.124&	33.126\\
        \hline
        2.700&	35.721&	35.737\\
        \hline
        2.800&	38.416&	38.417\\
        \hline
        2.900&	41.209&	41.209\\
        \hline
        3.000&	44.100&	44.087\\
        \hline
        3.100&	47.089&	47.090\\
        \hline
        3.200&	50.176&	50.163\\
        \hline
        3.300&	53.361&	53.357\\
        \hline
        3.400&	56.644&	56.651\\
        \hline
        3.500&	60.025&	60.023\\
        \hline
        3.600&	63.504&	63.489\\
        \hline
        3.700&	67.081&	67.080\\
        \hline
        3.800&	70.756&	70.757\\
        \hline
        3.900&	74.529&	74.537\\
        \hline
        4.000&	78.400&	78.394\\
        \hline
        4.100&	82.369&	82.376\\
        \hline
        4.200&	86.436&	86.440\\
        \hline
        4.300&	90.601&	90.605\\
        \hline
        4.400&	94.864&	94.868\\
        \hline
        4.500&	99.225&	99.213\\
        \hline
\end{tabular}
\caption{Wynik pomiarów}
\label{tab:my_label1}
\end{table}

\begin{table}[b]
\centering    
\begin{tabular}{|c|c|c|}
        \hline
        $t[s]$ & $s[m]$ & $s+\Delta s[m]$ \\
        \hline
        4.600&	103.684&	103.705\\
        \hline
        4.700&	108.241&	108.245\\
        \hline
        4.800&	112.896&	112.901\\
        \hline
        4.900&	117.649&	117.653\\
        \hline
        5.000&	122.500&	122.506\\    
        \hline
        5.100&	127.449&	127.458\\
        \hline
        5.200&	132.496&	132.490\\
        \hline
        5.300&	137.641&	137.649\\
        \hline
        5.400&	142.884&	142.885\\
        \hline
        5.500&	148.225&	148.213\\
        \hline
        5.600&	153.664&	153.667\\
        \hline
        5.700&	159.201&	159.180\\
        \hline
        5.800&	164.836&	164.818\\
        \hline
        5.900&	170.569&	170.558\\
        \hline
        6.000&	176.400&	176.385\\
        \hline
        6.100&	182.329&	182.352\\
        \hline
        6.200&	188.356&    188.345\\
        \hline
        6.300&	194.481&	194.488\\
        \hline
        6.400&	200.704&	200.705\\
        \hline
        6.500&	207.025&	207.026\\
        \hline
        6.600&	213.444&	213.441\\
        \hline
        6.700&	219.961&	219.965\\
        \hline
        6.800&	226.576&	226.582\\
        \hline
        6.900&	233.289&	233.290\\
        \hline
        7.000&	240.100&	240.102\\
        \hline
        7.100&	247.009&	247.017\\
        \hline
        7.200&	254.016&	254.018\\
        \hline
        7.300&	261.121&	261.127\\
        \hline
        7.400&	268.324&	268.333\\
        \hline
        7.500&	275.625&	275.607\\
        \hline
        7.600&	283.024&	283.040\\
        \hline
        7.700&	290.521&	290.529\\
        \hline
        7.800&	298.116&	298.105\\
        \hline
        7.900&	305.809&	305.796\\
        \hline
        8.000&	313.600&	313.605\\
        \hline
        8.100&	321.489&	321.480\\
        \hline
        8.200&	329.476&	329.472\\
        \hline
        8.300&	337.561&	337.571\\
        \hline
        8.400&	345.744&	345.735\\
        \hline
        8.500&	354.025&	354.038\\
        \hline
        8.600&	362.404&	362.401\\
        \hline
        8.700&	370.881&	370.879\\
        \hline
        8.800&	379.456&	379.451\\
        \hline
        8.900&	388.129&	388.135\\
        \hline
        9.000&	396.900&	396.895\\
        \hline
\end{tabular}
\caption{Wynik pomiarów (kontynuacja tabeli \ref{tab:my_label1})}
\label{tab:my_label2}
\end{table}

\begin{table}[h]
    \centering
    \begin{tabular}{|c|c|c|}
        \hline
        $t[s]$ & $s[m]$ & $s+\Delta s[m]$ \\
        \hline
        9.100&	405.769&	405.778\\
        \hline
        9.200&	414.736&	414.733\\
        \hline
        9.300&	423.801&	423.804\\
        \hline
        9.400&	432.964&	432.951\\
        \hline
        9.500&	442.225&	442.228\\
        \hline
        9.600&	451.584&	451.568\\
        \hline
        9.700&	461.041&	461.053\\
        \hline
        9.800&	470.596&	470.598\\
        \hline
        9.900&	480.249&	480.268\\
        \hline
        10.000&	490.000&	490.023\\ 
        \hline
    \end{tabular}
    \caption{Wynik pomiarów (kontynuacja tabeli \ref{tab:my_label2})}
    \label{tab:my_label3}
\end{table}

\begin{tikzpicture}
\begin{axis}[
    enlargelimits=false,
    axis lines = left,
    error bars/y dir=both,
    error bars/y explicit,
    xlabel = {\(t[s]\)},
    ylabel = {\(s[m]\)},
]
\addplot+[
    only marks,
    mark=oplus*,
    mark size = 1.2pt,
    % error bar legend,
    % error bars/.cd,
    error bar legend,
    error bars/.cd, 
    error mark options={gray},
    y dir=both,
    % mark=oplus*,
    % mark size=1pt
    ]
table[x=t, y=S+deltaS, y error=BLAD]
{dane_spadek_swobodny_2.csv};
\addlegendentry{\(s(t) + \Delta s\)}

\addplot[color=red,
domain = 0:10,
samples=100,
]{4.9*x^2};
\addlegendentry{\(s(t) = \frac{g*t^2}{2}\)}
\end{axis}
\end{tikzpicture}

% \begin{tikzpicture}
% \begin{axis}

% \end{axis}
% \end{tikzpicture}

Z pomiarów, podanych na tabelach \ref{tab:my_label1}, \ref{tab:my_label2}, \ref{tab:my_label3} wynika, że przyspieszenie ziemskie wynosi: $g=9,81  m⁄s^2$ .

\section{Wniosek}

Po eksperymentu mogę stwierdzić, że wartość przyspieszenia ziemskiego, otrzymanego w trakcie eksperymentu, równa się z wartością teoretyczną przyspieszenia ziemskiego.

\end{document}


\usepackage{graphicx}
\graphicspath{ {./images/} }
\usepackage[T1]{fontenc}
\usepackage{booktabs}

\begin{document}

\maketitle

\section{Wprowadzenie teoretyczne}

Dla ciała, spadającego swobodnie, kinematyczne równanie będzie przedstawione w następnej postaci:

\begin{equation}\label{Kinematyka}
    h(t) = h_0 + \frac{gt^2}{2}
\end{equation}

gdzie $h_0$ – początkowa wysokość, $g$ - przyspieszenie ziemskie.

Z równania \eqref{Kinematyka} wynika, że:
\begin{equation}\label{Przyspieszenie}
    g = \frac{2h_0}{t^2}
\end{equation}
Bo po spadnięciu ciała, wysokość w ostatnim momencie będzie równać się 0.

\section{Opis eksperymentu}

\begin{figure}[h]
    \centering
    \includegraphics{spadek}
    \caption{Przykład spadka swobodnego}
    \label{fig:rys1}
\end{figure}

Jak widać na rysunku \ref{fig:rys1}, będziemy wyrzucać ciało ze stałej wysokości kilku razy i zmierzać czas spadania. Znając początkową wysokość i czas spadania, możemy łatwo obliczyć przyspieszenie ziemskie.

\section{Wyniki pomiarów}

\begin{table}[b]
\centering
\begin{tabular}{|c|c|c|}
        \hline
        $t[s]$ & $s[m]$ & $s+\Delta s[m]$ \\
        \hline
             0&     0&  0.016\\
        \hline
        0.100&	0.049&	0.052\\
        \hline
        0.200&	0.196&	0.196\\
        \hline
        0.300&	0.441&	0.440\\
        \hline
        0.400&	0.784&	0.796\\
        \hline
        0.500&	1.225&	1.235\\
        \hline
        0.600&	1.764&	1.753\\
        \hline
        0.700&	2.401&	2.389\\
        \hline
        0.800&	3.136&	3.126\\
        \hline
        0.900&	3.969&	3.982\\
        \hline
        1.000&	4.900&	4.891\\
        \hline
        1.100&	5.929&	5.911\\
        \hline
        1.200&	7.056&	7.055\\
        \hline
        1.300&	8.281&	8.291\\
        \hline
        1.400&	9.604&	9.610\\
        \hline
        1.500&	11.025&	11.022\\
        \hline
        1.600&	12.544&	12.551\\
        \hline
        1.700&	14.161&	14.156\\
        \hline
        1.800&	15.876&	15.867\\
        \hline
        1.900&	17.689&	17.680\\
        \hline
        2.000&	19.600&	19.602\\
        \hline
        2.100&	21.609&	21.605\\
        \hline
        2.200&	23.716&	23.705\\
        \hline
        2.300&	25.921&	25.930\\
        \hline
        2.400&	28.224&	28.233\\
        \hline
        2.500&	30.625&	30.630\\
        \hline
        2.600&	33.124&	33.126\\
        \hline
        2.700&	35.721&	35.737\\
        \hline
        2.800&	38.416&	38.417\\
        \hline
        2.900&	41.209&	41.209\\
        \hline
        3.000&	44.100&	44.087\\
        \hline
        3.100&	47.089&	47.090\\
        \hline
        3.200&	50.176&	50.163\\
        \hline
        3.300&	53.361&	53.357\\
        \hline
        3.400&	56.644&	56.651\\
        \hline
        3.500&	60.025&	60.023\\
        \hline
        3.600&	63.504&	63.489\\
        \hline
        3.700&	67.081&	67.080\\
        \hline
        3.800&	70.756&	70.757\\
        \hline
        3.900&	74.529&	74.537\\
        \hline
        4.000&	78.400&	78.394\\
        \hline
        4.100&	82.369&	82.376\\
        \hline
        4.200&	86.436&	86.440\\
        \hline
        4.300&	90.601&	90.605\\
        \hline
        4.400&	94.864&	94.868\\
        \hline
        4.500&	99.225&	99.213\\
        \hline
\end{tabular}
\caption{Wynik pomiarów}
\label{tab:my_label1}
\end{table}

\begin{table}[b]
\centering    
\begin{tabular}{|c|c|c|}
        \hline
        $t[s]$ & $s[m]$ & $s+\Delta s[m]$ \\
        \hline
        4.600&	103.684&	103.705\\
        \hline
        4.700&	108.241&	108.245\\
        \hline
        4.800&	112.896&	112.901\\
        \hline
        4.900&	117.649&	117.653\\
        \hline
        5.000&	122.500&	122.506\\    
        \hline
        5.100&	127.449&	127.458\\
        \hline
        5.200&	132.496&	132.490\\
        \hline
        5.300&	137.641&	137.649\\
        \hline
        5.400&	142.884&	142.885\\
        \hline
        5.500&	148.225&	148.213\\
        \hline
        5.600&	153.664&	153.667\\
        \hline
        5.700&	159.201&	159.180\\
        \hline
        5.800&	164.836&	164.818\\
        \hline
        5.900&	170.569&	170.558\\
        \hline
        6.000&	176.400&	176.385\\
        \hline
        6.100&	182.329&	182.352\\
        \hline
        6.200&	188.356&    188.345\\
        \hline
        6.300&	194.481&	194.488\\
        \hline
        6.400&	200.704&	200.705\\
        \hline
        6.500&	207.025&	207.026\\
        \hline
        6.600&	213.444&	213.441\\
        \hline
        6.700&	219.961&	219.965\\
        \hline
        6.800&	226.576&	226.582\\
        \hline
        6.900&	233.289&	233.290\\
        \hline
        7.000&	240.100&	240.102\\
        \hline
        7.100&	247.009&	247.017\\
        \hline
        7.200&	254.016&	254.018\\
        \hline
        7.300&	261.121&	261.127\\
        \hline
        7.400&	268.324&	268.333\\
        \hline
        7.500&	275.625&	275.607\\
        \hline
        7.600&	283.024&	283.040\\
        \hline
        7.700&	290.521&	290.529\\
        \hline
        7.800&	298.116&	298.105\\
        \hline
        7.900&	305.809&	305.796\\
        \hline
        8.000&	313.600&	313.605\\
        \hline
        8.100&	321.489&	321.480\\
        \hline
        8.200&	329.476&	329.472\\
        \hline
        8.300&	337.561&	337.571\\
        \hline
        8.400&	345.744&	345.735\\
        \hline
        8.500&	354.025&	354.038\\
        \hline
        8.600&	362.404&	362.401\\
        \hline
        8.700&	370.881&	370.879\\
        \hline
        8.800&	379.456&	379.451\\
        \hline
        8.900&	388.129&	388.135\\
        \hline
        9.000&	396.900&	396.895\\
        \hline
\end{tabular}
\caption{Wynik pomiarów (kontynuacja tabeli \ref{tab:my_label1})}
\label{tab:my_label2}
\end{table}

\begin{table}[h]
    \centering
    \begin{tabular}{|c|c|c|}
        \hline
        $t[s]$ & $s[m]$ & $s+\Delta s[m]$ \\
        \hline
        9.100&	405.769&	405.778\\
        \hline
        9.200&	414.736&	414.733\\
        \hline
        9.300&	423.801&	423.804\\
        \hline
        9.400&	432.964&	432.951\\
        \hline
        9.500&	442.225&	442.228\\
        \hline
        9.600&	451.584&	451.568\\
        \hline
        9.700&	461.041&	461.053\\
        \hline
        9.800&	470.596&	470.598\\
        \hline
        9.900&	480.249&	480.268\\
        \hline
        10.000&	490.000&	490.023\\ 
        \hline
    \end{tabular}
    \caption{Wynik pomiarów (kontynuacja tabeli \ref{tab:my_label2})}
    \label{tab:my_label3}
\end{table}

Z pomiarów, podanych na tabelach \ref{tab:my_label1}, \ref{tab:my_label2}, \ref{tab:my_label3} wynika, że przyspieszenie ziemskie wynosi: $g=9,81  m⁄s^2$ .

\section{Wniosek}

Po eksperymentu mogę stwierdzić, że wartość przyspieszenia ziemskiego, otrzymanego w trakcie eksperymentu, równa się z wartością teoretyczną przyspieszenia ziemskiego.

\end{document}
